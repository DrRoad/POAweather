% Created 2012-10-11 Thu 14:13
\documentclass[a4paper]{article}
\usepackage[utf8]{inputenc}
\usepackage{hyperref}
\usepackage{graphicx}
\usepackage{longtable}
\usepackage{float}
\providecommand{\alert}[1]{\textbf{#1}}

\title{POAweather Ding 2012}
\author{Ivan Hanigan}
\date{\today}
\hypersetup{
  pdfkeywords={},
  pdfsubject={},
  pdfcreator={Emacs Org-mode version 7.8.11}}

\begin{document}

\maketitle

% Org-mode is exporting headings to 3 levels.
\tableofcontents
\hrule
\section{Intro}
\label{sec-1}

This is a data extraction from the POAweather project for Ning Ding, NCEPH staff member.
\section{This document is an executable script}
\label{sec-2}

This document is produced from an R script that mixes computer code with narrative.

The structure of this document follows the - \emph{Reichian load, clean, func, do} approach \href{http://stackoverflow.com/a/1434424}{http://stackoverflow.com/a/1434424} first put forward by Josh Reich.  
The workflow is implemented here using the ProjectTemplate package \href{http://projecttemplate.net/}{http://projecttemplate.net/} by John Myles White.
\section{Citation Requirements}
\label{sec-3}

Use of these data is open to all staff and students at NCEPH however do require the citations in the Reference list be cited in all publications.

The POAweather project should be cited.  It is a combination of code and data produced from the original paper by Hanigan, Hall and Dear in 2006 \cite{Hanigan2006} which compared 5 simple methods for estimating exposure to weather variables for populations of small areas (postcodes).  The updated source codes are available from \cite{Hanigan2012d}.

The source data are from the BoM \cite{NationalClimateCentreoftheBureauofMeteorology2010} and the ABS \cite{AustralianBureauofStatistics2006} and are hosted at the National Centre for Epidemiology and Population Health of The Australian National University  (using a PostgreSQL database \href{http://www.postgresql.org}{http://www.postgresql.org} with the PostGIS spatial extension \href{http://postgis.refractions.net}{http://postgis.refractions.net}).
\section{Authorship Requirements}
\label{sec-4}

These data must only be used by projects that produce NCEPH output, i.e. authorship and/or grant funding where an NCEPH staff member is a major participant.
\section{Statement of Compliance}
\label{sec-5}



\begin{center}
\begin{tabular}{ll}
 Details                                  &  User  \\
\hline
 Name:                                    &        \\
 Organisation:                            &        \\
 I agree to abide by these requirements:  &        \\
 Date:                                    &        \\
\hline
\end{tabular}
\end{center}
\section{The Codes}
\label{sec-6}
\subsection{main.r}
\label{sec-6-1}

This file is used to run the load, clean, func and do modules.  It is found in the root of the project directory.
\subsection{load.r}
\label{sec-6-2}
\subsection{clean.r}
\label{sec-6-3}
\subsection{func.r}
\label{sec-6-4}
\subsubsection{lib}
\label{sec-6-4-1}
\subsubsection{connect2postgres}
\label{sec-6-4-2}
\subsubsection{postIDW}
\label{sec-6-4-3}
\subsubsection{weathervars}
\label{sec-6-4-4}
\subsection{do.r}
\label{sec-6-5}
\subsubsection{do-prototype.r}
\label{sec-6-5-1}
\subsubsection{do-final-run}
\label{sec-6-5-2}
\section{Conclusion}
\label{sec-7}


\section{References}
\begin{thebibliography}{1}

\bibitem{Hanigan2006}
Ivan Hanigan, Gillian Hall, and Keith Dear.
\newblock {A comparison of methods for calculating population exposure
  estimates of daily weather for health research.}
\newblock {\em International journal of health geographics}, 5(1):38, 2006.

\bibitem{Hanigan2012d}
Ivan~C. Hanigan.
\newblock {POAweather. [[https://github.com/ivanhanigan/POAweather][https://github.com/ivanhanigan/POAweather]]}, 2012.

\bibitem{NationalClimateCentreoftheBureauofMeteorology2010}
{National Climate Centre of the Bureau of Meteorology}.
\newblock {\em {Daily or three hourly weather data for Bureau of Meteorology
  stations.}}
\newblock 700 Collins Street Docklands VIC 3008, AUSTRALIA;, 2010.

\bibitem{AustralianBureauofStatistics2006}
{Australian Bureau of Statistics}.
\newblock {2923.0.30.001 - Census of Population and Housing: Census Geographic
  Areas Digital Boundaries, Australia}.
\newblock [[http://www.abs.gov.au/AUSSTATS/abs@.nsf/DetailsPage/2923.0.30.0012006?OpenDocument][http://www.abs.gov.au/AUSSTATS/abs@.nsf/DetailsPage/2923.0.30.0012006?OpenDocument]] 2006.


\end{thebibliography}

\end{document}